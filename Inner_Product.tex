\documentclass[a4paper]{article}
\usepackage{amsmath}
\usepackage{amssymb}
\usepackage[margin=2cm]{geometry}
\usepackage{amsthm}
\usepackage{cite}
\usepackage{hyperref}
\usepackage{physics}
\newcommand{\norm}[1]{\lvert \lvert#1\rvert \rvert}
\newcommand{\inner}[1]{\langle #1 \rangle}
\title{\textbf{Inner Product Spaces and Normed Vector Spaces}\\(A004)}
\author{Paul Valentine\\ Email: \href{mailto:paul@the-valentines.co.uk}{paul@the-valentines.co.uk}}
\date{\today}
\begin{document}
\maketitle
\begin{abstract}
  A vector space over a feild endowed with additional informatoin on how to take two vectors to the field enables a concept of length and angles to be defined in the vector space. This addition information is the inner product. This paper presents the basic concepts surrounding the inner product on a vector space. Extending from this a normed vector space is discussed representing a concepth of length.
\end{abstract}
\section{Definition and Axioms}
We can define an \textbf{inner product space} as a vector space $V$ over a field $F$ that has an inner product defined. The inner product is a map:
\begin{equation}
  \langle \cdot,\cdot \rangle : V \times V \to F
\end{equation}
This map must satisfy three properties $\forall v \in V$
\begin{align}
  \text{conjugate symmetry:  }&\langle a, b \rangle = \overline{\langle b,a \rangle}\\
  \text{Linearity:  }&\alpha \langle a,b \rangle = \langle \alpha a,b \rangle, \: \: \alpha \in F\\
  &\langle a+b,c \rangle = \langle a,c \rangle + \langle b,c \rangle\\
  \text{Positive definite:  } &\langle a,a \rangle > 0 \: \: \forall a \in V \backslash \vec{0}
\end{align}
\subsection{Eucledian Space}
\begin{equation}
  a,b \in \mathbb{R}^n, \: \: \inner{a,b} = \sum_{i=1}^n a_i b_i
\end{equation}
This is a special case of an inner product in $\mathbb{R}^n$ and is called the dot product. Here  $\inner{a,b} = a \cdot b$.
\\ \\
Consider a unit vector $\vec{e}_x$ along the x-axis of a Eucledian plan and $\vec{e}_y$ along the y axis. Here we know that $\vec{e}_i \cdot \vec{e}_j = \delta_i^j$. Rotate $\vec{e}_y$ to an angle $\theta$ form $\vec{e}_x$ to form a new set of basis vectors $\vec{e}_r$ and $\vec{e}_s$.
\begin{equation}
  x = r cos(\theta) \qquad y = r sin(\theta)
\end{equation}
Under these coordinate transformations the new basis vectors are:
\begin{align}
  &\vec{e}_r = \pdv{}{r} rcos(\theta) \vec{e}_x + \pdv{}{r} rsin(\theta) \vec{e}_y = cos(\theta)\vec{e}_x + sin(\theta)\vec{e}_y\label{e1}\\
  &\vec{e}_s = \vec{e}_x\label{e2}
\end{align}
Consdier a vector $V$ pointing along $\vec{e}_s$ and vector $W$ pointing along $\vec{e}_r$ such that $\norm{V} = V^s$ and $\norm{W}=W^r$.
\begin{align}
  V\cdot W = V^s \vec{e}_s \cdot W^r \vec{e}_r = V^s W^r (\vec{e}_s \cdot \vec{e}_r)\label{e3}
\end{align}
From equation \ref{e1} and equation \ref{e2} $\vec{e}_s \cdot \vec{e}_r = cos(\theta)$. Applying this result to equation \ref{e3} yields:
\begin{equation}
  \label{e4}
  V.W=\norm{V} \norm{W} \:cos({\theta})
\end{equation}
Equation \ref{e4} says $V \cdot V = \norm{V}^2$ and that for $V,W \ne 0 \: V\cdot W = 0 \implies V$ and $W$ are perpendicular vectors. This definition can be carried into other inner product spaces.
\section{Normed Vector Spaces}
A normed vector space $V$ over a field $F$ is space have a nonnegative valued function, a norm, $p:V \to F$. $p$ has the following axioms.
\begin{align}
  p(\vec{a}+\vec{b}) \le p(\vec{a})+p(\vec{b})\\
  p(\alpha \vec{a}) = \lvert \alpha \rvert p(\vec{a})\\
  p(\vec{a})=0 \implies \vec{v} = \vec{0}\label{posdef}
\end{align}
If a function satisfies the above with the acception of equation \ref{posdef} then it is a seminorm. The notation for a norm may be $\norm{\vec{a}}$. Not all normed vector spaces are an inner product space. 
\subsection{Eucledian Norm}
The may be referred to as the $L^2$ norm. $L^p$ is a function space defined with a $p$-norm for a finite dimension.
\begin{equation}
  \norm{\vec{x}}_2 = \sqrt{\sum_{i=1}^n (x_i)^2} = \sqrt{\vec{x}\cdot \vec{x}}
\end{equation}
\subsection{Manhatton norm}
This is the $L^1$ norm.
\begin{equation}
  \norm{x}_1 = \sum_{i=1}^n \rvert x_i \lvert
\end{equation}
This is simply the absolute sum of the components of the vector.
\end{document}
