\documentclass[a4paper]{article}
\usepackage{amsmath}
\usepackage{amssymb}
\usepackage[margin=2cm]{geometry}
\usepackage{amsthm}
\title{\textbf{Cartesian Products}\\(A002)}
\author{Paul Valentine}
\date{18 March 2020}
\begin{document}
\maketitle
\begin{abstract}%TODO: need to rewrite the cartesian product abstract
  Cartesian products defin ordered sets of elements. Their relevance crops up in studies of topology and topological manifolds in physics. To describe motion a curve in the manifold $M$ is paramaterised by $\lambda \in \mathbb{R} \mid \gamma : \mathbb{R} \to M , \exists x:M \to \mathbb{R}^n \implies x \circ \gamma : \mathbb{R} \to \mathbb{R}^n$. Where such coorindate maps exist on a topological space we can define mathematics by conidering only the mapping acros the Reals. In doing so the concept of space arises which is formed from the cartesian product of base space with, for example, a tangent space. This requires a basic understanding of the definition of a cartesian product.
\end{abstract}
\section{Cartesian Products}
\subsection{General}
Let A and B be two sets. The cartesian product is defined by:
\begin{equation}
A \times B = \{(a,b)\mid a\in A, b\in B\}
\end{equation}
It is the multiplicatoin of two sets to form a set of \textbf{ordered} pairs. For example if:
\begin{align*}
&A=\{ jo, pip \}\\
&B = \{car, house\}\\
&A\times B = \{(jo,car),(jo,house),(pip,car),(pip,house)\}
\end{align*}
The pair is ordered becase
\begin{align*}
  \{(jo,car),(jo,house),(pip,car),(pip,house)\} \neq \{(car,jo),(house,jo),(car,pip),(house,pip)\}
\end{align*}
A practical exmaple is to let $X$ be the set of points on the $x$ line and $Y$ be the set of points on the $y$ line. Then $X \times Y$ prepresents the points on the $XY$ plane.\\ \\
We construct a tuple of any number of elements from $n$ number of $\mathbb{R}$:
\begin{equation}
\underbrace{\mathbb{R} \times \mathbb{R} \times \cdots \times \mathbb{R}}_{\text{n times}}=\mathbb{R}^n
\end{equation}
\subsection{Empty Sets}
The result of multiplying by the empty set is the empty set.
\begin{align}
  \mathbb{R} \times \varnothing = \varnothing
\end{align}
\subsection{Non-commutativity and non-associativity}
\begin{equation}
  A \times B \neq B \times A
\end{equation}
Unless $A=B$ or either $A$ or $B$ is the empty set. For example $\{1\} \times \{2\} = (1,2) \neq \{2\} \times \{1\} = (2,1)$ because of the ordering of the pairs.
\begin{equation}
  (A \times B) \times C \neq A \times (B \times C)
\end{equation}
Unless $A$, $B$ or $C =\varnothing$.\\ \\
$\square$
\end{document}
