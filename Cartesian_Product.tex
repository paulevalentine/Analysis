\documentclass[a4paper]{article}
\usepackage{amsmath}
\usepackage{amssymb}
\usepackage[margin=2cm]{geometry}
\title{\textbf{Cartesian Products}\\(A002)}
\author{Paul Valentine}
\date{18 March 2020}
\begin{document}
\maketitle
\section{Cartesian Products}
\subsection{General}
Let A and B be two sets. The cartesian product is defined by:
\begin{equation}
AxB = {(a,b)\mid a\in A, b\in B}
\end{equation}
It is the multiplicatoin of two sets to form a set of \textbf{ordered} pairs. For example if:
\begin{align*}
&A=\{ jo, pip \}\\
&B = \{car, house\}\\
&A\times B = \{(jo,car),(jo,house),(pip,car),(pip,house)\}
\end{align*}
A practical exmaple is to let $X$ be the set of points on the $x$ line and $Y$ be the set of points on the $y$ line. Then $X \times Y$ prepresents the points on the $XY$ plane.\\ \\
We can therefore say for $n$ number of $\mathbb{R}$:
\begin{equation}
\underbrace{\mathbb{R} \times \mathbb{R} \times \cdots \times \mathbb{R}}_{\text{n times}}=\mathbb{R}^n
\end{equation}
\subsection{Empty Sets}
The result of multiplying by the empty set is the empty set.
\begin{align}
  \mathbb{R} \times \varnothing = \varnothing
\end{align}
\subsection{Non-commutativity and non-associativity}
\begin{equation}
  A \times B \neq B \times A
\end{equation}
Unless $A=B$ or either $A$ or $B$ is the empty set.
\begin{equation}
  (A \times B) \times C \neq A \times (B \times C)
\end{equation}
Unless $A$, $B$ or $C =\varnothing$.
\end{document}
