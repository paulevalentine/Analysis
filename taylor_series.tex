\documentclass[a4paper]{article}
\usepackage{amsmath}
\usepackage{amssymb}
\usepackage[margin=2cm]{geometry}

\title{\textbf{Taylor Series Expansions}}
\author{Paul Valentine}
\begin{document}
\maketitle
\begin{abstract}
Taylor series expands a function about a point $a$ as a polynomial. This is useful as an analytical tool to make approximations about the function at a point. 
\end{abstract}
\section{Derivation of the single variable case}

Consider estimate a function $f(x)$ local to the point $f(a)$. Take the linear approximation.
\begin{align}
&f(x) = f(a)+f'(a)(x-a)
\end{align}
To improve accuracy consider a polynomial expansion 
\begin{equation}
\label{eq1}
f(x)=\sum_{n=0}^{n=\infty}	c_n (x-a)^n
\end{equation}
The problem then becomes how to calculate the coefficients. To test a solution differentiate an expansion in a set number of terms and then differentiate. 
\begin{align}
&f(x) = c_0 + c_1 (x-a) + c_2 (x-a)^2 + c_3 (x-a)^3 + c_4 (x-a)^4\cdots\label{eq2}\\
&f'(x) = c_1 + 2 c_2 (x-a) + 3c_3 (x-a)^2 +4c_4 (x-a)^3 \cdots\\
&f''(x) = 2 c_2 + 6c_3(x-a) + 12c_4(x-a)^2\cdots\\
&f'''(x)=6c_3 + 24(x-a) \cdots
\end{align}
By setting $x=a$ in each of the above we see that we isolate the coefficient that is the degree of the derivative.
\begin{align}
&c_0 = f(a)\\
&c_1 = f'(a)\\
&c_2 = \frac{f''(a)}{2!}\\
&c_3 = \frac{f'''(a)}{3!}\cdots\; c_n =\frac{f^{(n)}(a)}{n!}
\end{align}
Using these into equation \ref{eq2} we have
\begin{equation}
\label{eq3}
f(x) = f(a) + f'(a)(x-a) + \frac{f''(a)}{2!}(x-a)^2+\frac{f'''(a)}{3!}(x-a)^3 \dot{....}
\end{equation}
Equation \ref{eq3} can be recast in the form of equation \ref{eq1}.
\begin{equation}
f(x)=\sum_{n=0}^{n=\infty} \frac{f^{(n)}(a)}{n!} (x-a)^n
\end{equation}
If the series is summed to a finite value $k$ then this only represents an approximation. We must therefore add an additional function $h_k(x)(x-a)^k$ which is a remainder that must have the form $\displaystyle{\lim_{x\to a}}h_k(x)=0$ such that:
\begin{equation}
f(x)=\sum_{n=0}^{n=k} \frac{f^{(n)}(a)}{n!} (x-a)^n +h_k(x)(x-a)^k
\end{equation}
Nothing is stated in this paper about calculating the remainder.
\section{Multi variable version}
\end{document}