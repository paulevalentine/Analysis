\documentclass[a4paper]{article}
\usepackage{amsmath}
\usepackage{amssymb}
\usepackage[margin=2cm]{geometry}
\usepackage{amsthm}
\usepackage{cite}
\title{\textbf{Vector Spaces}\\(A003)}
\author{Paul Valentine}
\date{18 March 2020}
\begin{document}
\maketitle
\section{Introduction}
A vector space is a collection of objects called \textbf{vectors} that can be scaled and added together in a linear way. It is therefore a linear space. If $a \in \mathbb{R}$ then a real vector space is defined. If $a \in \mathbb{C}$ then a complex vector space exists. The scaling and addition rules for a vector space form a set of axioms. Roughly speaking, the number of independent directoins is the dimension of the vector space.
\section{Operations}
\label{operations}
\begin{equation}
  \label{add}
  +:V \times V \to V
\end{equation}
Equation \ref{add} defines the addition rule where $\times$ is the cartesian production\cite{A002} where $V$ is a set. With a field $F$ we have.
\begin{equation}
  \cdot : F \times V \to V
\end{equation}
Where $\cdot$ represents scalar multiplication. If $F$ is a complex field then we have a complex vector space. If $F$ is a real number field then we have a real vector space.
\begin{align}
  \text{Associativity of addition: } &\hat{u}+(\hat{v}+\hat{w}) = (\hat{u}+\hat{v})+\hat{w}\\
  \text{Commutativity of addition: }&\hat{u}+\hat{v} = \hat{v}+\hat{u}
\end{align}
\section{Axioms}
For the set $V$ to form a vector space the operations $+$ and $\cdot$ must conform to the following axioms:
\bibliography{bibrefs}
\bibliographystyle{ieeetr}
\end{document}
