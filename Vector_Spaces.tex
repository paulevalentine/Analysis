\documentclass[a4paper]{article}
\usepackage{amsmath}
\usepackage{amssymb}
\usepackage[margin=2cm]{geometry}
\usepackage{amsthm}
\usepackage{cite}
\title{\textbf{Vector Spaces}\\(A003)}
\author{Paul Valentine}
\date{18 March 2020}
\begin{document}
\maketitle
\section{Introduction}
A vector space is a collection of objects called \textbf{vectors} that can be scaled and added together in a linear way. It is therefore a linear space. The scaling and addition rules for a vector space form a set of axioms. Roughly speaking, the number of independent directoins is the dimension of the vector space. Two operations are defined on this space $\cdot$ and $+$. This paper describes the basics of vector spaces.
\section{Operations $\cdot$ and $+$}
Let $V$ be an ordered set.
\label{operations}
\begin{equation}
  \label{add}
  +:V \times V \to V
\end{equation}
Equation \ref{add} defines the addition rule where $\times$ is the cartesian production\cite{A002} and $V$ is a set. For example.
\begin{equation}
  +:\big((x_1, y_1),(x_2,y_2)\big) = \big((x_1+x_2,y_1+y_2)\big)
\end{equation}
 With a \textbf{field}\cite{wiki:mathfield} $F$ we have.
\begin{equation}
  \cdot : F \times V \to V
\end{equation}
Where $\cdot$ represents scalar multiplication. If $F$ is a complex field then we have a complex vector space. If $F$ is a real number field then we have a real vector space. A field is a set on which addition, subtraction, multiplication and division are defined. That it is, it is an albelian group with $0$ as the additive identity and $1$ as the multiplicative identity. Example:
\begin{equation}
\forall a\in \mathbb{R}  \cdot:\big(a,(x_1,y_1)\big) = \big((ax_1,ay_1)\big)
\end{equation}
Examples of field are $\mathbb{R}, \mathbb{C}, \mathbb{Z} \text{ and } \mathbb{Q}$.\\ \\
Coupling the above we say that a vector space V is over a field F.
\section{Axioms}
For the set $V$ to form a vector space the operations $+$ and $\cdot$ must conform to the following 8 axioms:
\begin{align}
  \text{Associativity of addition: } &\vec{u}+(\vec{v}+\vec{w}) = (\vec{u}+\vec{v})+\vec{w}\\
  \text{Commutativity of addition: }&\hat{u}+\hat{v} = \vec{v}+\vec{u}\\
  \text{Identity element of addition: }&\exists\:\vec{0}\in V \mid \hat{v} + \vec{0} = \hat{v} \:\forall \: \hat{v}\in V\\
  \text{Inverse element of addition: } &\forall \: \vec{v}\in V\: \exists \: -\vec{v} \mid \vec{v} + -\vec{v}=\vec{0}\\
  \text{Compatibility of scalar and field multiplication: }& a(b \vec{v}) = (ab) \vec{v}\\
  \text{Identity element of scale multiplication: }& 1 \vec{v} = \vec{v}\\
  \text{Distributivity of scalar multiplicatoin w.r.t vector addition: } &a(\vec{v}+\vec{u})= a\vec{v}+a\vec{u}\\
  \text{Ditributivity of scalar multiplication w.r.t field addition: }&(a+b)\vec{v} = a\vec{v}+b\vec{v}
\end{align}
\section{Examples}
\subsection{Coordinates spaces}
The simplest form a vector space is a field itself giving n tuples $(a_1, a_2, \dots, a_n)$.
\subsection{Function Space}
Let V be a vector space over a field F and X be any set. The set $f:X \to V$ are a vector space over F. Typically here $F,V \in \mathbb{R}$ as a common example.
\begin{align}
  &f,g:X \to V:\forall x \in X, \forall a \in F\\
  &(f+g)(x)=f(x)+g(x)\\
  &(a\cdot f)(x)=a \cdot f(x)
\end{align}
If X is also a vector space the set of linear maps $X \to V$ form a vector space over F called $Hom(X,V)$. A special case is the set of linear functions $V \to F$ which form a \textbf{dual} vector space.\\ \\
\subsection{Integration \cite{wiki:integration}}
The collectoin of integrable functions form a vector space.
\begin{align}
  &f \mapsto \int_a^b f(x) dx\\
  &\int_a^b (\alpha f + \beta g)(x) dx = \int_a^b (\alpha f(x)+\beta g(x)) dx=\alpha \int_a^b f(x)dx + \beta \int_a^b g(x)dx
\end{align}
Since the function itself forms a vector space $f(x) dx$ takes the function to the field over which the integration occurs. Since this is $I:V \to F$ it places $f(x)dx$ $(I)$ in the dual vector space.
\\ \\
$\square$
\bibliography{bibrefs}
\bibliographystyle{ieeetr}
\end{document}
